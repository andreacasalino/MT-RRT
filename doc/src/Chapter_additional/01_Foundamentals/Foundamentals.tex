\chapter{Foundamental concepts}

\section{What is an RRT algorithm?}

Rapidly Random exploring Tree(s), aka RRT(s), is one of the most popular technique adopted for solving planning path problems in robotics.
In essence, a planning problem consists of finding a feasible trajectory or path that leads a manipulator, or more in general a dynamical system, from a starting configuration/state to an ending desired one, consistently with a series of constraints.
RRTs were firstly proposed in \cite{RRT_LaValle}. They are able to explore a state space in an incremental way, building a search tree, even if they may require lots of iterations before terminating. They were proved be capable of always finding at least one solution to a planning problem, if a solution exists, i.e. they are probabilistic complete.
RRT were also proved to perform well as kinodynamic planners, designing optimal LQR controllers driving a generic dynamical system to a desired final state, see \cite{LQR_RRT_01} and  \cite{LQR_RRT_02}.
\\
The typical disadvantage of RRTs is that for medium-complex problems, they require thousands of iterations to get the solution. 
For this reason, the aim of this library is to provide multi-threaded planners implementing parallel version of RRTs, in order to speed up the planning process. 
\\
It is possible to use this library for solving any problem handled by an RRT algorithm. The only necessary thing to do when facing a new class of problem is to derive some specific objects describing the problem itself as detailed in Section \ref{chap:custom}.
\\
At the same time, one of the most common problem to solve with RRT is a standard path planning for an articulated arm.
What matters in such cases is to have a collision checker, which is not provided by this library. Anyway, the interfaces Tunneled$\_$check$\_$collision and Bubbles$\_$free$\_$ configuration allows you to integrate the collision checker you prefer for solving standard path planning problems (see also Section \ref{sec:steer_articulated}).
\\
\\
The next Section briefly reviews the basic mechanism of the RRT. The notations and formalisms introduced in the next Section will be also adopted by the other Sections. Therefore, the reader is strongly encouraged to read before the next Section.
\\
Section \ref{sec:pipeline} will describe the typical pipeline to consider when using MT-RRT, while some examples of planning problems are reported in Chapter \ref{chap:custom}. Chapter \ref{chap:parallel_RRT} will describe the possible parallelization strategy that MT-RRT offers you.
\footnote{A similar guide, but in a html format, is also available at \url{http://www.andreacasalino.altervista.org/__MT_RRT_doxy_guide/index.html}.}.

\section{Background on RRT}

\subsection{Standard RRT}
\label{sec:RRT_single}

RRTs explore the state space of a particular problem, in order to find a series of states connecting a starting $x_o$ and an ending one $x_f$, at the same time accounting for the presence of constraints. More precisely, this is done by building at least a search tree $T(x_o)$ having $x_o$ as root. Each node $x_i \in T$ is connected to its unique father $x_{fi}=Fath(x_f)$ by a trajectory $\tau_{fi \rightarrow i}$.
The root $x_o$ is the only node not having a father ($Fath(x_o)=\emptyset$). 
The set $\mathcal{X} \subseteq \mathbb{R}^{d} $ will contain all the possible states $x$ of the system whose motion must be controlled, while $\underline{\mathcal{X}} \subseteq \mathcal{X}$ is a subset describing the admissible region induced by a series of constraints. 
The solution we are interested in, consists clearly of a sequence of trajectories $\tau$ entirely contained in $\underline{\mathcal{X}}$.
\\
If we consider classical path planning problems, the constraints are represented by the obstacles populating the scene, which must be avoided.
However, according to the nature of the problem to solve, different kind of constraints might need to be accounted.
The basic version of an RRT algorithm is described by Algorithm \ref{alg:RRT_single}, whose steps are visually represented by Figure \ref{fig:RRT_single}.
Essentially, the tree is randomly grown by performing several steering operations. Sometimes, the extension of the tree toward the target state $x_f$ is tried in order to find an edge leading to that state.

\begin{algorithm}
 \caption{Standard RRT. A deterministic bias is introduced for connecting the tree toward the specific target state $x_f$.
 The probability $\sigma$ regulates the frequency adopted for trying the deterministic extension. The Extension procedure is described in algorithm \ref{alg:Extend}.
 \label{alg:RRT_single}}
 \KwData{$x_o$, $x_f$}
 $T= \lbrace x_o \rbrace$\; 
 \For{$k=1:MAX\_ITERATIONS$}{
 sample $r \sim U(0,1)$\;
 \If{$r < \sigma$}{
 $x_{steered}=$Extend$(T, x_f)$\;
 \If{$x_{steered}$ is $VALID$}{
 \If{$\Vert x_{steered} - x_f \Vert \leq \epsilon$}{
 \textbf{Return} Path$\_$to$\_$root($x_{steered}$)$ \cup x_f$\;
 }
 }
 }
 \Else{
 sample a $x_R \in \mathcal{X}$\;
 Extend$(T, x_R)$\;
 }
 }
 \end{algorithm}

\begin{algorithm}
 \caption{The Extend procedure. 
 \label{alg:Extend}}
 \KwData{$T$, $x_R$}
 $x_{Nearest}=$Nearest$\_$Neighbour$(T, x_R)$\;
 $x_{steered}=$ Steer($x_{Nearest}, x_R$)\;
\If{$x_{steered} \notin \underline{\mathcal{X}}$}{
	Mark $x_{steered}$ as $INVALID$\;
} 
 \If{$x_{steered}$ is $VALID$}{
 $T=T \cup x_{steered}$ \;
 }
 \textbf{Return} $x_{steered}$\;
 \end{algorithm}

\begin{algorithm}
 \caption{The Nearest$\_$Neighbour procedure: the node in $T$ closest to the given state $x_R$ is searched.
\label{alg:NearNeigh}}
\KwData{$T$, $x_R$}
\textbf{Return} $\underset{x_i \in T}{\operatorname{argmin}}( C(\tau_{i \rightarrow R } ) )$;
\end{algorithm}

The Steer function in algorithm \ref{alg:Extend} must be problem dependent. Basically, It has the aim to extend a certain state $x_i$ already inserted in the tree, toward another one $x_R$. To this purpose, an optimal trajectory $\tau_{ i \rightarrow R}$, agnostic of the constraints, going from $x_i$ to $x_R$, must be taken into account. Ideally, the steering procedure should find the furthest state from $x_i$ that lies on  $\tau_{ i \rightarrow R}$ and for which the portion of $\tau_{ i \rightarrow R}$ leading to that state is entirely contained in 
$\underline{\mathcal{X}}$. However, in real implementations the steered state returned might be not the possible farthest from $x_i$. Indeed, the aim is just to extend the tree toward $x_R$. At the same time, in case such the steered state results too closer to $x_i$, the steering should fails \footnote{This is done to avoid inserting less informative nodes in the tree, reducing the tree size.}.

 \begin{figure}
	 \centering
 \def\svgwidth{0.45 \columnwidth}
 \import{../src/Chapter_additional/01_Foundamentals/image/}{steer.pdf_tex} 
	 \caption{The dashed curves in both pictures are the optimal trajectories, agnostic of the constraints, connecting the pair of states $x_{Near}$ and $x_f$, while the filled areas are regions of $X$ not allowed by constraints.
 The steering procedure is ideally in charge of searching the furthest state to $x_{Near}$ along $\tau_{ Near \rightarrow f}$. For the example on the right, the steering is not possible: the furthest state along $\tau_{ Near \rightarrow f}$ is too much closer to $x_{Near}$.}
 \label{fig:Steer}
 \end{figure}


The Nearest$\_$Neighbour procedure relies on the definition of a cost function $C(\tau)$. Therefore, the closeness of states does not take into account the shape of $\underline{\mathcal{X}}$. Indeed $C(\tau)$ it's just an estimate agnostic of the constraints. Then, the constraints are taken into account when steering the tree.
The algorithm terminates when a steered configuration $x_s$ sufficiently close to $x_f$ is found.     
\\
The steps involved in the standard RRT are summarized by Figure \ref{fig:RRT_single}.

 \begin{figure}
	 \centering
 \def\svgwidth{0.85 \columnwidth}
 \import{../src/Chapter_additional/01_Foundamentals/image/}{RRT_iterations.pdf_tex} 
	 \caption{Examples of iterations done by an RRT algorithm. The solution found is the one connecting the state in the tree that reached $x_f$, with the root $x_o$.}
 \label{fig:RRT_single}
 \end{figure}


\subsection{Bidirectional version of the RRT}
\label{sec:RRT_bidir}

The behaviour of the RRT can be modified leading to a bidirectional strategy \cite{RRT_bid}, which expands simultaneously two different trees. Indeed, at every iteration one of the two trees is extended toward a random state. Then, the other tree is extended toward the steered state previously obtained. At the next iteration, the roles of the trees are inverted. The algorithm stops, when the two trees meet each other. The detailed pseudocode is reported in Algorithm \ref{alg:RRT_bid}.

\begin{algorithm}
 \caption{Bidirectional RRT. A deterministic bias is introduced for accelerating the steps.
 The probability $\sigma$ regulates the frequency adopted for trying the deterministic extension. 
 The Revert procedure behaves as exposed in Figure \ref{fig:RRTBid_single}.
 \label{alg:RRT_bid}}
 \KwData{$x_o$, $x_f$}
 $T_A = \lbrace x_o \rbrace$\; 
 $T_B = \lbrace x_f \rbrace$\; 
 $x_{target} = $ root of $T_A$\;
 $x_2 = $ root of $T_B$\;
 $T_{master} = T_A$\; 
 $T_{slave} = T_B$\;
 \For{$k=1:MAX\_ITERATIONS$}{
 sample $r \sim U(0,1)$\;
 \If{$r < \sigma$}{  $x_{steered}=$Extend$(T_{master}, x_{target})$\; }
 \Else{
 sample a $x_R \in \mathcal{X}$\;
 $x_{steered}=$Extend$(T_{master}, x_R)$\;
 }
 \If{$x_{steered}$ is $VALID$}{
 $x_{steered2}=$Extend$(T_{slave}, x_{steered})$\;
 \If{$x_{steered2}$ is $VALID$}{
 \If{$\Vert x_{steered} - x_{steered2} \Vert \leq \epsilon$}{
 \textbf{Return} Path$\_$to$\_$root($x_{steered}$) $\cup$ Revert $\bigg($ Path$\_$to$\_$root($x_{steered2}$) $\bigg)$ \;
 }
 }
 }
 Swap $T_{target}$ and $T_2$ \;
 Swap $T_{master}$ and $T_{slave}$ \;
 }
\end{algorithm}


 This solution offers several advantages. For instance, the computational times absorbed by the Nearest Neighbour search is reduced since this operation is done separately for the two trees and each tree contains at an average half of the states computed. The steps involved in the bidirectional strategy are depicted in Figure \ref{fig:RRTBid_single}.

 \begin{figure}
	 \centering
 \def\svgwidth{0.85 \columnwidth}
 \import{../src/Chapter_additional/01_Foundamentals/image/}{RRTBid_iterations.pdf_tex} 
	 \caption{Examples of iterations done by the bidirectional version of the RRT. The path in the tree rooted at $x_f$ is reverted to get the second part of the solution.}
 \label{fig:RRTBid_single}
 \end{figure}
 

\subsection{Compute the optimal solution: the RRT*}
\label{sec:RRT_star}

For any planning problem there are infinite $\tau_{o \rightarrow f} \subset \underline{ \mathcal{X} }$, i.e. infinite trajectories starting from $x_o$ and terminating in $x_f$ which are entirely contained in the admissible region $\underline{ \mathcal{X} }$. Among the aforementioned set, we might be interested in finding the trajectory minimizing the cost $C(\tau_{o \rightarrow f})$, refer to Figure \ref{fig:path_costs}.
The basic version of the RRT algorithm is proved to find with a probability equal to 1, a suboptimal solution \cite{RRT_star}.
The optimality is addressed by a variant of the RRT, called RRT* \cite{RRT_star}, whose pseudocode is contained in Algorithm \ref{alg:RRT_star}.
Essentially, the RRT* after inserting in a tree a steered state, tries to undertake local improvements to the connectivity of the tree, in order to minimize the cost-to-go of the states in the $Near$ set. This approach is proved to converge to the optimal solution after performing an infinite number of iterations \footnote{In real cases, after a sufficient big number of iterations an optimizing effect can be yet appreciated. }. 
There are no precise stopping criteria for the RRT*: the more iterations are performed, the more the solution found get closer to the optimal one.

 \begin{figure}
	 \centering
 \def\svgwidth{0.4 \columnwidth}
 \import{../src/Chapter_additional/01_Foundamentals/image/}{path_costs.pdf_tex} 
	 \caption{Different trajectories connecting $x_o$ with $x_f$, entirely contained in $\underline{\mathcal{X}}$. If we assume as cost the length of a path, the red solution is the optimal one.}
 \label{fig:path_costs}
 \end{figure}


 \begin{algorithm}
 \caption{RRT*. The Extend$\_$Star, Rewird and Cost$\_$to$\_$root procedures are explained in, respectively, algorithm \ref{alg:Expand_star}, \ref{alg:rewird} and \ref{alg:cost_to_root}.
 \label{alg:RRT_star}}
 \KwData{$x_o$, $x_f$}
 $T= \lbrace x_o \rbrace$\; 
 $Solutions = \emptyset$\;
 \For{$k=1:MAX\_ITERATIONS$}{
 sample $r \sim U(0,1)$\;
 \If{$r < \sigma$}{
 $x_{steered}=$Extend$\_$Star$(T, x_f)$\;
 \If{$x_{steered}$ is $VALID$}{
 \If{$\Vert x_{steered} - x_f \Vert \leq \epsilon$}{
 $Solutions = Solutions \cup x_{steered}$ \;
 }
 }
 }
 \Else{
 sample a $x_R \in \mathcal{X}$\;
 Extend$\_$Star$(T, x_R)$\;
 }
 }
 $x_{best} = \underset{x_S \in Solutions}{\operatorname{argmin}}$( Cost$\_$to$\_$root($x_S$) )\;
 \textbf{Return} Path$\_$to$\_$root$(x_{best}) \cup x_f$;
 \end{algorithm}

 \begin{algorithm}
 \caption{The Extend$\_$Star procedure.  $d$ is the cardinality of $\mathcal{X}$.
 \label{alg:Expand_star}}
 \KwData{$T$, $x_R$}
 $x_{steered}$ = Extend($T$, $x_R$)\;
 \If{$x_{steered}$ is $VALID$}{
 $Near=\bigg \lbrace x_i \in T \bigg| C(\tau_{i \rightarrow steered}) \leq \gamma {(\frac{log(\left | T \right |)}{\left | T \right |})}^{\frac{1}{d}}  \bigg \rbrace$ \;
 Rewird($Near, x_{steered}$)\;
 }
 \textbf{Return} $x_{steered}$\;
 \end{algorithm}

 \begin{algorithm}
 \caption{The Rewird procedure.
 \label{alg:rewird}}
 \KwData{$Near, x_s$}
 $x_{best father} = Fath(x_s)$\;
 $C_{min} = C(\tau_{best father \rightarrow s})$\;
 \For{$x_n \in Near$}{
 \If{$\tau_{n \rightarrow s} \subset \underline{\mathcal{X}}$ AND $C(\tau_{n \rightarrow s}) < C_{min}$}{
 $C_{min}=C(\tau_{n \rightarrow s})$\; 
 $x_{best fath}=x_n$\;
 }
 }
 $Fath(x_s) = x_{best fath}$\;
 $C_s=$Cost$\_$to$\_$root($x_s$)\;
 $Near = Near \setminus x_{best fath}$\;
 \For{$x_n \in Near$}{
 \If{$\tau_{s \rightarrow n} \subset \underline{\mathcal{X}}$}{
 $C_n=C(\tau_{s \rightarrow n}) + C_s$\;
 \If{$C_n <$ Cost$\_$to$\_$root($x_n$)}{
 $Fath(x_n)=x_s$\;
 }
 }
 }
 \end{algorithm}

 \begin{algorithm}
 \caption{The Cost$\_$to$\_$root procedure computing the cost spent to go from the root of the tree to the passed node.
 \label{alg:cost_to_root}}
 \KwData{$x_n$}
 \If{$Fath(x_n) = \emptyset$}{
 \textbf{Return} 0\;
 }
 \Else{
 \textbf{Return} $C(\tau_{Fath(n) \rightarrow n}) + $ Cost$\_$to$\_$root($Fath(x_n)$) \;
 }
 \end{algorithm}

\section{MT-RRT pipeline}
\label{sec:pipeline}

 \begin{figure}
	 \centering
 \def\svgwidth{0.9 \columnwidth}
 \import{../src/Chapter_additional/01_Foundamentals/image/}{pipeline.pdf_tex} 
	 \caption{Pipeline to follow for using MT-RRT.}
 \label{fig:pipeline}
 \end{figure}

When solving a planning problem with MT-RRT, the pipeline of Figure \ref{fig:pipeline} should be followed.

\subsubsection{A) Define the problem}
\label{sec:define_problem}

Whenever you need to solve a new class of planning problem, you should instantiate a Problem object.
This class should contains all the information describing the particular problem to solve (how to compute the optimal trajectories $\tau$, the shape of the constraints admitted region $\underline{\mathcal{X}}$, etc..) and its made of 2 main components:
\begin{itemize}
\item sampler, i.e. a Sampler object drawing random states to allow the tree(s) growth.
\item trajManager, i.e. the TrajectoryFactory able to evaluate optimal trajectories $\tau$ connecting pairs of states.
\end{itemize}
The 2 above objects should be built and passed to Problem constructor. 
Chapter \ref{chap:custom} reports some examples of planning problems, describing how to derive the corresponding Sampler and TrajectoryFactory.

\subsubsection{B) Instantiate a Solver}

After having defined the problem, you need to build a Solver object, to be used to solve one or more problems of a certain kind.
Every time a new problem is solved calling Solver::solve(...), the relevant information are stored inside this object and can be accessed before solving a new one.
Clearly, only the results concerning the last found solution are saved, i.e. data are overwritten when calling multiple times Solver::call(...), but with different starting and ending configurations and possibly different trees obtained.

\subsubsection{C) Set the strategy}

Before solving any problem you should also specify to the Solver the strategy that you would like to use. In this case, the strategy pattern is adopted and a Strategy needs to be built. In this case you don't have to derive your own strategy, but you can use one of the objects already defined in the core package, namely:
\begin{itemize}
\item SerialStrategy
\item QueryParallStrategy
\item SharedTreeStrategy
\item LinkedTreesStrategy
\item MultiAgentStrategy
\end{itemize}
Each Strategy support all the rrt algorithms described in Sections \ref{alg:RRT_single}, \ref{sec:RRT_bidir} and \ref{alg:RRT_star}, with the only exception that MultiAgentStrategy does not support the bidirectional algorithm.
It is always possible to extract the strategy from a Solver, set up the parameters in a different way and then re-assign the strategy to the Solver.
\\
QueryParallStrategy, SharedTreeStrategy, LinkedTreesStrategy and MultiAgentStrategy are multi-threaded strategies and more details are provided Sections \ref{sec:strtg_query}, \ref{sec:strtg_shared}, \ref{sec:strtg_copied} and \ref{sec:strtg_multi}. On the opposite SerialStrategy implements the standard mono-thread versions of rrt, i.e. the ones detailed in the previous Sections.


\subsubsection{D) Access the results after solving the problem}

There two main ways to access the results stored inside Solver:
\begin{itemize}
\item get the found solution, i.e. a series of states $x_{1,2,3,\hdots,M}$ that must be visited to get from the starting configuration to the ending one, by traversing the trajectories $\tau_{1 \rightarrow 2 } , \tau_{2 \rightarrow 3}, \hdots ,\tau_{M-1 \rightarrow M} \subset \underline{\mathcal{X}}$. Such series of waypoints can be obtained by calling Solver::copyLastSolution. In case a solution was not found, an empty collection is returned.
\item get the tree(s) computed for finding the solution. This is done using Solver::extractLastTrees. Beware that by default the trees are not saved and stored after solving a new problem, bit this behaviour can be enabled calling Strategy::saveTreesAfterSolve.
Moreover, calling this Solver::extractLastTrees, remove the trees from Solver, moving them to the outside.
\end{itemize}
Additional information (the iterations spent, the computation time) regarding the solution can be accessed using the others getters of Solver.
